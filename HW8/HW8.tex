
\documentclass[12pt]{article}
\title{CISC 102 (Fall 20)\\ Homework \#8: Counting $\;$   (24 Points) }
\author{Student Name/ID:. . . . . . . .}
\date{}


\usepackage[left=2cm, right=3cm, top=1cm]{geometry}

\usepackage{amsmath}
\usepackage{amsfonts}
\usepackage{amssymb}
\usepackage{amsthm}
\usepackage{bm}						% for \bm -- bold almost everywhere
\usepackage{mdwlist}					% for {itemize*}
\usepackage{enumerate}
\usepackage{xcolor,graphicx}
\usepackage{relsize,etoolbox}
\usepackage{tikz}

\newcommand{\abs}[1]{\left| #1 \right|}
\newcommand{\ab}[1]{\left[ #1 \right]}
\newcommand{\rb}[1]{\left( #1 \right)}
\newcommand{\set}[1]{\left\{ #1 \right\}}
\newcommand{\norm}[1]{\left\| #1 \right\|}
\newcommand{\wt}[1]{\widetilde{#1}}
\newcommand{\pdif}[2]{\frac{\partial #1 }{\partial #2}}
\newcommand{\pdifm}[3]{\frac{\partial^{#3} #1 }{\partial #2^{#3}}}

\renewcommand{\a}{\ensuremath{\mathcal{A}}}
\renewcommand{\b}{\ensuremath{\mathcal{B}}}
\newcommand{\bb}{\ensuremath{\mathbb{B}}}
\newcommand{\ee}{\ensuremath{\mathbb{E}}}
\newcommand{\f}{\ensuremath{\mathcal{F}}}
\newcommand{\g}{\ensuremath{\mathcal{G}}}
\newcommand{\h}{\ensuremath{\mathcal{H}}}
\renewcommand{\l}{\ensuremath{\mathcal{L}}}
\newcommand{\map}{\longrightarrow}
\newcommand{\nn}{\ensuremath{\mathbb{N}}}
\newcommand{\one}{\ensuremath{\mathbf{1}}}
\newcommand{\p}{\ensuremath{\mathcal{P}}}
\newcommand{\pp}{\ensuremath{\mathbb{P}}}
\newcommand{\qq}{\ensuremath{\mathbb{Q}}}
\newcommand{\rr}{\ensuremath{\mathbb{R}}}
\newcommand{\zz}{\ensuremath{\mathbb{Z}}}

\let\oldemptyset\emptyset
\let\emptyset\varnothing

\newcommand{\var}{\mathrm{var}}

\DeclareMathOperator{\argmax}{argmax}

\begin{document}

\maketitle

\begin{enumerate}

\item (2 pts) \\
How many ways are there to select a 5 card poker hand from a standard deck of 52 cards, such that none of the cards are clubs?\\
\\We know that clubs make up $\frac{1}{4}$ of the deck, so there are $52\times \frac{3}{4}=39$ cards that are not clubs.
\\$C(39,5)= \frac{39!}{5!34!} = 575757$
\\$\therefore$ there are 575757 ways to select a 5 poker hand such that none of the cards are clubs.

\item (2pts) \\
How many ways are there to select a 5 card poker hand  from a standard deck of 52 cards, such that at least one of the cards is a club?\\
\\In this example we know that we have 13 clubs to choose from, and there are still 39 remaining that are not clubs so we have
\\$13 \times \binom{39}{4}$
\\For a hand involving $i$ clubs we have $\binom{13}{i} \times \binom{39}{5-i}$
\\$\sum_{5}^{i} \binom{13}{i} \times \binom{39}{5-i}$ (in the next step calculating directly)
\\=$\binom{52}{5} - \binom{39}{5} = \frac{52!}{5!47!} - \frac{39!}{5!34!} = 2023203$
\\$\therefore$ there are 2023203 ways to choose a hand of 5 cards with at least one club in it.

\item (6 pts)\\
You are planning a dinner party and want to choose 5 people to attend from a list of 11 close personal friends.
\begin{enumerate}
	\item In how many ways can you select the 5 people to invite.\\
    \\There are $\frac{11!}{5!6!} = 462$ ways of choosing 5 people to invite.
		
	\item Suppose two of your friends are a couple and will not attend unless the other is invited. How many different ways can you invite 5 people under these constraints?\\
    \\In this example, we have two cases, one where we invite the couple and choose 3 friends of 9 which we can do with $\binom{9}{3}$ ways. The second case is we don't invite the couple and choose 5 friends out of the 9 remaining which we can do in $\binom{9}{5}$ ways. So we have:
    \\$\binom{9}{3} + \binom{9}{5} = 210$
    \\$\therefore$ there are 210 ways to choose who to invite when two of the friends are a couple and won't go if the other is not invited.

	\item Suppose two of your friends are enemies, and will not attend unless the other is not invited. How many different ways can you invite 5 people under these constraints?\\
    \\We know that we have a total of $\binom{11}{5} = \frac{11}{5!6!} = 462$ ways to form invitations. The number of invitations that contain the enemies is $\binom{9}{3} = \frac{9!}{3!6!} = 84$. So know we know that there are $462 - 84 = 378$ possible invitations to not include the enemies.
\end{enumerate}


\item (2 pts) \\
	Prove that for \(n \geq 2,   \ 2 \cdot \binom{n}{2} + \binom{n}{1}=n^{2}\)\\
    \\Proving through induction:
    \\Basis Step:
    \\$P(2)$ holds since $2 \times \binom{2}{2} + \binom{2}{1} = 2^2$
    \\$\rightarrow 2 \times \frac{2!}{2!0!} + \frac{2!}{1!1!} = 4$
    \\$\rightarrow 2 + 2 = 4 \rightarrow 4 = 4$
    \\$\therefore P(2)$ holds so basis step complete.
    \\Inductive step:
    \\Inductive hypothesis that $P(k)$ is true for integers equal or greater than 2, so our statement is $2 \times \binom{k}{2} + \binom{k}{1} = k^2$. If $P(k)$ is true, then $P(k+1)$ must also be true.
    \\$\rightarrow 2 \times \binom{k+1}{2} + \binom{k+1}{1} = (k+1)^2$
    \\$\rightarrow 2 \times \frac{(k+1)!}{2!(k-1)!} + \frac{(k+1)!}{1!k!} = k^2$
    \\$\rightarrow \frac{(k+1)(k)(k-1)!}{(k-1)!} + \frac{(k+1)k!}{k!} = k^2 + 2k + 1$
    \\$\rightarrow k(k+1) + (k+ 1) = k^2 + 2k + 1$
    \\$\rightarrow k^2 + 2k + 1 = k^2 + 2k + 1$
    \\$\therefore$ This shows that if the hypothesis $P(k)$ is true then $P(k+1)$ must also be true.
    \\This shows since the basis and inductive steps have been completed, so $P(n)$ is true for all values of $n$ such that it is greater or equal to 2. $\therefore n \geq 2,   \ 2 \cdot \binom{n}{2} + \binom{n}{1}=n^{2}$ is indeed true.


\item (2 pts)\\
Let \(S = \lbrace1,2,3,...n\rbrace\) where {\it  n }  is even. What is the relationship between the number of subsets {\it  S }  of with odd cardinality, and the number of subsets of {\it  S }  with even cardinality. \\
Explain your answer.
\leavevmode\\\relax
Hint: Each subset either contains the element 1 or it doesn't. Consider all the subsets that do not contain the element 1. Some of these have odd cardinality and some have even cardinality. What happens when you include the element 1 in each of these subsets?\\
\\Let subsets of $S$ be $s$
\\A subset $s \subseteq S$ with an even number of elements is determined by its intersection with $\{x_1,...,n\}$: if the intersection has an even number of elements then $x_n \notin S$, and if it has an odd number of elements $x_n \in S$. $\therefore$ the number of subsets of S with an even number of elements is equal to the number of subsets of odd elements is $2^{n-1}$




\item (2 pts)
Lotto 6/49 lets you pay \$3 for the thrill of choosing six different numbers from the set \(\lbrace1,2,3, \dots 49  \rbrace\).   Rumour has it that there may (on extremely rare occasions) be other benefits to buying a lottery ticket.\\
How many ways are there to choose the six numbers?  Work out the exact value.\\
\\There are $\binom{49}{6} = \frac{49!}{6!(49-6)!} = 13983816$ ways of choosing 6 numbers from the set of 49 available.


\item (2 pts) \\
Prove that \[\sum_{i=1}^{n} \binom{i}{2}= \binom{n+1}{3} \quad \forall \thinspace n \geq 2\]\\
\\Proving through induction:
\\Basis step:
\\$\sum_{i=1}^{2} \binom{i}{2}= \binom{2+1}{3}$
\\$P(2)$ holds since, $\rightarrow \binom{1}{2} + \binom{2}{2}= \binom{3}{3} \rightarrow 0 + \frac{2!}{2!0!}=\frac{3!}{3!0!} \rightarrow 1 = 1$
\\$\therefore P(2)$ holds so basis step complete.
\\Inductive step:
\\The inductive hypothesis is that $P(k)$ holds for any value of $k$ greater or equal to 2, so $\sum_{i=1}^{k} \binom{i}{2}= \binom{k+1}{3}$. If $P(k)$ is true, then $P(k+1)$ must also be true. To show that, add $\binom{k+1}{2}$ to both sides.
\\$\binom{1}{2} + \binom{2}{2}+...+\binom{k}{2} + \binom{k+1}{2}=\binom{k+1}{3}+ \binom{k+1}{2}$
\\$\rightarrow \binom{1}{2} + ... +\binom{k}{2}  + \binom{k+1}{2} = \binom{k+1}{3} + \binom{k+1}{2}$
\\$\rightarrow \binom{k+1}{3} + \binom{k+1}{2} = \binom{k+2}{3}$ (We get this from pascals identity)
\\$\therefore$this shows that if $P(k)$ is true, then $P(k+1)$ must also be true. Since both the basis step and the inductive step have been completed, $P(n)$ must be true for all values of $n$ that are equal to or greater than 2. $\therefore \sum_{i=1}^{n} \binom{i}{2}= \binom{n+1}{3} \quad \forall \thinspace n \geq 2$ is true.


\item (2 pts) \\ What is the number of ways to colour $n$ different objects, one colour per object with 2 colours? What is the number of ways to colour $n$ different objects with 2 colours, so that each colour is used at least once.\\
    \\The number of ways to color $n$ objects with 3 colors is $3^n$, and now we subtract the colorings which did not use any of the three colors. The number of ways to color with just 2 colors is $2^n$ and since there are three colors that can be used we have $3^n - 3(2)^n$ but he have now removed all colorings with a given color twice so we now need to add those back again and we get $3^n - 3(2)^n + 3$. $\therefore$ The number of ways to color $n$ different objects with 2 colors, so each is used once is $3^n - 3(2)^n + 3$.

\item (2pts)\\
Prove that

$$
{n \choose 0} - { n \choose 1} + {n \choose 2} - {n \choose 3} + \cdots  + (-1)^n{n \choose n} = 0
$$

Note that this equation can also be written as follows:

$$
\sum_{i=0}^n {n \choose i}(-1^{i}) = 0
$$
HINT: This can be viewed as a special case of the binomial theorem.\\
\\Using binomial theorem with $x=-1$ and $y=1$.
\\We se that $0=0^n=((-1)+1)^n = \sum_{i=0}^{n} \binom{n}{i}(-1)^{i}1^{n-i}$
\\$=\sum_{i=0}^{n} \binom{n}{i}(-1)^i$
\\$\therefore$ we see that $0=\sum_{i=0}^{n} \binom{n}{i}(-1)^i$




\item (2 pts)\\
 A class of 100 students must select  2 safety officers per week for 7 weeks. There are no constraints on how the selection is made, and the same student can be selected more that once. In how many ways can this selection be made?\\
 \\Selecting the safety officer can be represented using $\binom{100}{2} = 4950$ can be represented in a week, but since it happens once a week so we multiply 4950 by 7 and we get $4950 \times 7 = 34650$ ways. $\therefore$ there are 34650 ways of choosing 2 safety officers from a group of 100 students for 7 weeks.

\
\end{enumerate}


\vfill
\end{document}
