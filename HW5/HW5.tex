\documentclass[12pt]{article}
\title{CISC 102 (Fall 20)\\ Homework \#5: Sequences,  Recursion \&  Induction  $\;$   (25 Points) }
\author{Student Name/ID:. . . . . . . .}
\date{}


\usepackage[left=2cm, right=3cm, top=1cm]{geometry}

\usepackage{amsmath}
\usepackage{amsfonts}
\usepackage{amssymb}
\usepackage{amsthm}
\usepackage{bm}						% for \bm -- bold almost everywhere
\usepackage{mdwlist}					% for {itemize*}
\usepackage{enumerate}
\usepackage{xcolor,graphicx}
\usepackage{relsize,etoolbox}
\usepackage{tikz}

\newcommand{\abs}[1]{\left| #1 \right|}
\newcommand{\ab}[1]{\left[ #1 \right]}
\newcommand{\rb}[1]{\left( #1 \right)}
\newcommand{\set}[1]{\left\{ #1 \right\}}
\newcommand{\norm}[1]{\left\| #1 \right\|}
\newcommand{\wt}[1]{\widetilde{#1}}
\newcommand{\pdif}[2]{\frac{\partial #1 }{\partial #2}}
\newcommand{\pdifm}[3]{\frac{\partial^{#3} #1 }{\partial #2^{#3}}}

\renewcommand{\a}{\ensuremath{\mathcal{A}}}
\renewcommand{\b}{\ensuremath{\mathcal{B}}}
\newcommand{\bb}{\ensuremath{\mathbb{B}}}
\newcommand{\ee}{\ensuremath{\mathbb{E}}}
\newcommand{\f}{\ensuremath{\mathcal{F}}}
\newcommand{\g}{\ensuremath{\mathcal{G}}}
\newcommand{\h}{\ensuremath{\mathcal{H}}}
\renewcommand{\l}{\ensuremath{\mathcal{L}}}
\newcommand{\map}{\longrightarrow}
\newcommand{\nn}{\ensuremath{\mathbb{N}}}
\newcommand{\one}{\ensuremath{\mathbf{1}}}
\newcommand{\p}{\ensuremath{\mathcal{P}}}
\newcommand{\pp}{\ensuremath{\mathbb{P}}}
\newcommand{\qq}{\ensuremath{\mathbb{Q}}}
\newcommand{\rr}{\ensuremath{\mathbb{R}}}
\newcommand{\zz}{\ensuremath{\mathbb{Z}}}

\let\oldemptyset\emptyset
\let\emptyset\varnothing

\newcommand{\var}{\mathrm{var}}

\DeclareMathOperator{\argmax}{argmax}

\begin{document}

\maketitle

\par\noindent Solutions are due before 11:59 PM on \textbf{   November 1, 2020 }.


\begin{enumerate}

\item (2pts) \\
Prove by induction that \[\sum_{j=1}^n 2^j = 2^{n+1} - 2 ~, \forall \, n \ge 1\]
\\Basis Step:
\\$P(1)$ holds since $\sum_{j = 1}^{1} 2^1 = 2^1 = 2^{1+1} - 2 \rightarrow 2 = 2$\\
\\Inductive Step:
\\The inductive hypothesis is that $P(K)$ is true as an integer equal to or greater than 1, so our statement is $2^1 + 2^2 + 2^3 + ... + 2^k = 2^{k +1} - 2$. If $P(k)$ is true, then $P(k + 1)$ must also be true. To show its true we add $2^{k + 1}$ to both sides.
\\$2^1 + 2^2 + 2^3 + ... + 2^k + 2^{k + 1}= 2^{k+ 1} - 2 + 2^{k + 1}$
\\$2^{k + 1} - 2 + 2^{k + 1} = 2(2)^{k + 1} - 2$
\\$= 2^{k+2} - 2$
\\This also shows that if hypothesis $P(k)$, then $P(k + 1)$ must also be true.\\
\\This shows since the basis and inductive have been completed, so $P(n)$ is true for all nonnegative integers greater to or equal to 1. Shows the formula for the sum of the geometric sequence is indeed correct.

\item  (2pts) \\
Prove by induction:
\[\forall n \geq 1, (n^3-n) \text{ is divisible by 3 }\]
\\Basis Step:
\\P(1) is true since $P(1) = (1)^3 - (1) = 0$, and 0 is divisible by 3.\\
\\Inductive Step:
\\We assume that $p(k)$ holds for a positive integer $k$ so $k^3 - k$ is divisible by 3 for an arbitrary value of k, and it follows that $P(k + 1)$ is divisible by 3. $(k + 1)^3 - (k + 1)$. If we expand we get $(k^3 + 3k^2 + 3k + 1) - (k + 1)$, (we get rid of the ones but do not simplify for k). Then we rearrange as $(k^3 - k) + (3k^2 + 3k) = (k^3 - k) + 3(k^2 + k)$. We can now conclude that using our hypothesis that $k^3 - k$ is divisible by 3, and the second term is divisible by 3 since it is 3 times an integer.\\
\\Since we have completed both the basis and inductive steps, we proved that $n^3 - n$ is divisible by 3 for any positive integer.

\item (2pts) \\
Find a closed form for \[a_{1}=2, a_{n}=a_{n-1}+n+6\]
\\$a_{1} = 2$
\\$a_{2} = a_{1} + 2 + 6 = 2 + 2 + 6 = 10$
\\$a_{3} = a_{2} + 3 + 6 = 10 + 3 + 6 = 19$
\\$a_{4} = a_{3} + 4 + 6 = 19 + 4 + 6 = 29$
\\$a_{5} = a_{4} + 5 + 6 = 29 + 5 + 6 = 40$
\\$a_{6} = a_{5} + 6 + 6 = 40 + 6 + 6 = 52$
\\Closed formula: $a_{n} = \frac{1}{2}(n^2 + 13n - 10)$

\item (2pts) \\
Find a closed form for the recurrence relation \[a_n = a_{n-1} + 2n, \text{ with } a_1 = 2\]
Prove that your closed form is correct.
\\$a_{1} = 2$
\\$a_{2} = 2 + 2(2) = 6$
\\$a_{3} = 6 + 2(3) = 12$
\\$a_{4} = 12 + 2(4) = 20$
\\$a_{5} = 20 + 2(5) = 30$
\\$a_{6} = 30 + 2(6) = 42$
\\Closed formula: $a_{n} = n^2 + n$
\\Proof:
\\Case 1: $a_{1} = 1^2 + 1 = 2$, Case 2: $a_{3} = 3^2 + 3 = 12, Case 3: 6^2 + 6 = 42$
\\$\therefore$This closed formula works for all of the terms in this sequence.

\item (4pts)
\begin{enumerate}
	\item  Find a recurrence relation that defines the sequence 1, 1, 1, 1, 2, 3, 5, 9, 15, 26, ...
	\par (Hint: each number in the sequence is based on the four numbers just before it in the sequence.)
    \\$a_{n} = -(a_{n-4}) + a_{n-3} + a_{n-2} + a_{n-1}$ Where $a_{1}=a_{2}=a_{3}=a_{4}=1$
	\item Now find a different sequence that satisfies the recurrence relation you found in (a)
    \\1, 2, 3, 4, 8, 13, 22, 39, 66, 114
\end{enumerate}

\item (2pts)\\
Consider the following recurrence relation:
\[a_n = a_{n-1} + 2n, \text{ with } a_1 = 3\]
Prove by induction that
\[a_n = n^2 + n + 1 ~ \forall ~ n \geq 1\]
\\Basis Step:
\\$a_{1}$ is true since $a_{1} = (1)^2 + (1) + 1 = 3$\\
\\Inductive Step:
\\We assume that the value of $a_{k}$, where $k$ is a positive integer equal to or greater to 1, $a_{k} = k^2 + k + 1$. If $a_{k}$ is true, which is a solution to $a_{k} = a_{k-1} + 2k$ with $a_{1} = 3$. If this is true then $k+1$ must also be true.\\
\\$a_{k+1} = a_{k} + 2(k+1) = a_{k} + 2k + 2$ (First equation)
\\$a_{k+1} = (k+1)^2 + (k+1) + 1= k^2 + 2k + 1 + k + 1 + 1 = k^2 + 3k + 3$ (What we want)
\\(Sub in $a_{k}$ into the first equation)
\\$= (k^2 + k + 1) + 2k + 2 = k^2 + 3k + 3$
\\$\therefore$This proves that $a_{k+1}$ is true.\\
\\Since we have completed both basis and inductive steps we have proven that $a_n = n^2 + n + 1$ is the closed form for the recurrence relation $a_n = a_{n-1} + 2n$ for all values of n greater or equal to 1.

\item (2pts)\\
Consider the following recurrence relation:
\[a_n = 2 \cdot a_{n-1} - 3 \text{ with } a_1 = 5\]
Prove by induction that \[a_n = 2^n + 3 \ \forall ~ n \geq 1\]\\
\\Basis Step:
\\$a_{1}$ is true since $a_{1} = 2^1 + 3 = 5$.\\
\\Inductive Step:
\\We assume the value of $a_{k}$ where $k$, where $k \geq 1$, $a_{k} = 2^k + 3$ (first equation), which is a solution to $a_{k} = 2 a_{k-1} - 3$ (Second equation). If these statements of $k$ are true, then $k + 1$ must also be true.\\
\\$a_{k+1} = 2 a_{k} - 3$ (Third equation)
\\$a_{k+1} = 2^{k+1} + 3$ (Fourth equation: What we want)
\\Sub in first equation into third equation
\\$= 2(2^k + 3) - 3$
\\$= 2^{k+1} + 3$
\\$\therefore$This proves $a_{k+1}$ is true\\
\\Since we have proven both the basis and inductive steps we have proven that $a_n = 2^n + 3$ is the solution to $a_n = 2 \cdot a_{n-1} - 3$ for $n \geq 1$.

\item (4pts)\\
\begin{enumerate}
	\item  Find a closed-form solution for this recurrence relation:
	\[ a_n = 2 \cdot a_{n-1} - n + 1 \text{ with } a_1 = 2\]
    \\$a_{1} = 2$
    \\$a_{2} = 2(2) - 2 + 1 = 3$
    \\$a_{3} = 2(3) - 3 + 1 = 4$
    \\$a_{4} = 2(4) - 4 + 1 = 5$
    \\Closed formula: $a_{n} = 2n  - n + 1 = n + 1$
	\item  Prove that your closed-form solution is correct
    Case 1: $a_{1} = 1 + 1 = 2$, Case 2: $a_{2} = 2 + 1 = 3$, Case 3: $a_{3} = 3 + 1 = 4$, Etc...
\end{enumerate}

\item (5 pts) \\
Let $P(n)$ be the statement that a postage of $n$ cents can be
formed using just 4-cent stamps and 7-cent stamps. The
parts of this exercise outline a strong induction proof that
$P(n)$ is true for all integers $n \geq 18.$
\begin{enumerate}
	\item Show that the statements $P(18), P(19), P(20),$ and
	$P(21)$ are true, completing the basis step of a proof by
	strong induction that $P(n)$ is true for all integers $n \geq 18.$\\
    \\$P(18)$ can be formed by two 7-cent stamps and one 4-cent stamp ($2(7) + 4 = 18$). $P(19)$ can be formed by one 7-cent stamp and three 4-cent stamps ($7 + 3(4) = 19$). $P(20)$ can be formed by two five 4-cent stamps ($4(5) = 20$). $P(21)$ can be formed by three 7-cent stamps ($3(7) = 21$).\\

	\item What is the inductive hypothesis of a proof by strong
	induction that $P(n)$ is true for all integers $n \geq 18$?\\
    \\The inductive hypothesis states that $P(j)$ is true for $18 \leq j \leq k$, where $k$ is an integer with $k \geq 21$.\\
	\item What do you need to prove in the inductive step of a proof that P(n) is true for all integers $n \geq 18$?\\
    \\You need to prove that under the assumption above that $P(k+1)$ is true, to get $k + 1$ cents.\\
	\item Complete the inductive step for $k \geq 21.$\\
    \\Using the inductive hypothesis we can assume that $P(k - 3)$ is true since $k - 3 \geq 18$, where we can form $k - 3$ stamps using 7-cent and 4-cent stamps. To form $k + 1$ stamps, we need only add another 4-cent to the stamps used in $k - 3$ cents. From this we have shown the inductive hypothesis is true, so $P(k+1)$ is also true.\\
	\item Explain why these steps show that $P(n)$ is true for all integers $n \geq 18.$\\
    \\Since we have completed the basis step and the inductive step of this strong proof, we know by strong induction that $P(n)$ is true for all integers $n$ with $n \geq 12$. We know that every postage stamp of $n$ cents, where $n$ is at least 18, can be formed using 7-cent and 4-cent stamps.
\end{enumerate}
%%%%%%%%%%%%%%%%%%%%%%%%%%%%%%%%%%%%%%%%%%%%%%%%%%%%%%%%%%%%%%%%%%%%%%%%

%%%%%%%%%%%%
\end{enumerate}
\end{document} 