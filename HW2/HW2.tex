\documentclass[12pt]{article}
\title{CISC 102 (Fall 20)\\ Homework \#2: Logic $\;$   (25 Points) }
\author{Student Name/ID:. . . . . . . .}
\date{}


\usepackage[left=2cm, right=3cm, top=1cm]{geometry}

\usepackage{amsmath}
\usepackage{amsfonts}
\usepackage{amssymb}
\usepackage{amsthm}
\usepackage{bm}						% for \bm -- bold almost everywhere
\usepackage{mdwlist}					% for {itemize*}
\usepackage{enumerate}
\usepackage{xcolor}
\usepackage{relsize,etoolbox}

\newcommand{\abs}[1]{\left| #1 \right|}
\newcommand{\ab}[1]{\left[ #1 \right]}
\newcommand{\rb}[1]{\left( #1 \right)}
\newcommand{\set}[1]{\left\{ #1 \right\}}
\newcommand{\norm}[1]{\left\| #1 \right\|}
\newcommand{\wt}[1]{\widetilde{#1}}
\newcommand{\pdif}[2]{\frac{\partial #1 }{\partial #2}}
\newcommand{\pdifm}[3]{\frac{\partial^{#3} #1 }{\partial #2^{#3}}}

\renewcommand{\a}{\ensuremath{\mathcal{A}}}
\renewcommand{\b}{\ensuremath{\mathcal{B}}}
\newcommand{\bb}{\ensuremath{\mathbb{B}}}
\newcommand{\ee}{\ensuremath{\mathbb{E}}}
\newcommand{\f}{\ensuremath{\mathcal{F}}}
\newcommand{\g}{\ensuremath{\mathcal{G}}}
\newcommand{\h}{\ensuremath{\mathcal{H}}}
\renewcommand{\l}{\ensuremath{\mathcal{L}}}
\newcommand{\map}{\longrightarrow}
\newcommand{\nn}{\ensuremath{\mathbb{N}}}
\newcommand{\one}{\ensuremath{\mathbf{1}}}
\newcommand{\p}{\ensuremath{\mathcal{P}}}
\newcommand{\pp}{\ensuremath{\mathbb{P}}}
\newcommand{\qq}{\ensuremath{\mathbb{Q}}}
\newcommand{\rr}{\ensuremath{\mathbb{R}}}
\newcommand{\zz}{\ensuremath{\mathbb{Z}}}

\let\oldemptyset\emptyset
\let\emptyset\varnothing

\newcommand{\var}{\mathrm{var}}

\DeclareMathOperator{\argmax}{argmax}

\begin{document}

\maketitle

\par\noindent Solutions are due before 11:59 PM on \textbf{Friday  Midnight October 4, 2020 }.


\begin{enumerate}

\item (4 pts)
Let $p, q,$ and $r$ be the propositions\\
$p$: You have the flu, $q$: You miss the final examination, and $r$: You pass the course. \\
Express each of these propositions as an English sentence.
\begin{enumerate}
	\item 	$q \rightarrow  \neg r$
    \\If you miss the final exam, you won't pass the course.
	\item  $p \vee q \vee r$
    \\You have the flue, or you missed the final examination, or you pass the course
	\item $(p \rightarrow \neg r) \vee (q \rightarrow \neg r)$
    \\You have the flu so you don't pass the course, or you miss the final examination and won't pass the course.
	\item $(p \wedge q) \vee (\neg q \wedge r)$
    \\You have the flu and miss the final examination or you don't miss the final examination and pass the course.
\end{enumerate}	

\item (2 pts)
Write each of these statements in the form $\textrm{“if}\; p, \textrm{then}\; q”$
in English.
\begin{enumerate}
	\item I will remember to send you the address only if you send me an e-mail message.
\\If you send me an e-mail message, then I will remember to send you the address.\\
	
	\item The Red Wings will win the Stanley Cup if their 	goalie plays well.
\\If the Red Wings goalie plays well, then they will win the Stanley Cup.
\end{enumerate}

\item (3 pts)
State the converse, contrapositive, and inverse of each of
these conditional statements.
\begin{enumerate}

\item If it snows tonight, then I will stay at home.
\\Converse: If I stay home, then It will snow tonight.
\\Contrapositive: If I don't stay home, it will not snow tonight
\\Inverse: If it does not snow tonight, the I will not stay at home.\\

\item I go to the beach whenever it is a sunny summer day.
\\Converse: If it is a sunny summer day, then I will go to the beach
\\Contrapositive: If it is not a sunny summer day, then I will not go to the beach.
\\Inverse: I don't go to the beach whenever it is not a sunny summer day.\\

\item When I stay up late, it is necessary that I sleep until noon.
\\Converse: It is necessary that I sleep until noon, when I stay up late.
\\Contrapositive: It is not necessary that I sleep until noon, When I do not stay up late.
\\Inverse: When I don't stay up late, it is not necessary that I sleep until noon.
\end{enumerate}


\item (2 pts) Are the two logical expressions \(P \wedge \neg Q\) and \(\neg (\neg P \vee Q)\) logically equivalent? Answer this question using a truth table (or tables)
    \begin{displaymath}
    \begin{array}{|c|c|c|c|c|c|}
    P & Q & \lnot Q & P \land \lnot Q & \lnot P & \lnot (\lnot P \lor Q)\\
    \hline
      T & T & F & F & F & F\\
      T & F & T & T & F & T\\
      F & T & F & F & T & F\\
      F & F & T & F & T & F

    \end{array}
    \end{displaymath}
    \\ $\therefore$ We can see that both $P \land \lnot Q$ and $\lnot (\lnot P \lor Q)$ are both logically equivalent.

\item  (2 pts) Is \(\neg (P \rightarrow Q )\) logically equivalent to \(P \wedge \neg Q\)  ?
    \begin{displaymath}
    \begin{array}{|c|c|c|c|c|}
    P & Q & \lnot (P \rightarrow Q) & \lnot Q & P \land \lnot Q\\
    \hline
      T & T & F & F & F\\
      T & F & T & T & T\\
      F & T & F & F & F\\
      F & F & F & T & F

    \end{array}
    \end{displaymath}
    \\or
    \\ $\lnot (P \rightarrow Q) \equiv \lnot (\lnot P \lor Q)$
    \\$\equiv \lnot(\lnot P) \land \lnot Q$
    \\$\equiv P \land \lnot Q$
    \\$\therefore$ We can see that both $\lnot (P \rightarrow Q)$ and $P \land \lnot Q$ are both logically equivalent.
\item  (2 pts)
\begin{enumerate}
	\item  Prove that \(A \subseteq B \to A \cap \bar{B} = \emptyset\)
\\$A \subseteq B \equiv \forall x (x \in A \rightarrow x \in B)$
\\$\forall x (x \in A \rightarrow \lnot(x \in B))\equiv \forall x (x \in A \rightarrow x \not \in B)$
\\$\forall x (x \in A \rightarrow x \not \in B) \equiv \forall x (x \in A \rightarrow x \in \bar{B})$ (Which is false)
\\$\therefore \equiv \emptyset$

or we can use the complement laws since $A \subseteq B \rightarrow A \cap \bar{B} = \emptyset$. Since all the values of A also belong to B, and the compliment of B are all the elements of x "outside of B", therefore there is not intersection between $A$  and $\bar{B}$. And that is why a $\emptyset$ is returned as the output.\\
\item Prove that \(A \cap \bar{B} = \emptyset \to A \subseteq B\)
    \\$A \cap \bar{B} \equiv \forall x (x \in A \rightarrow x \in \bar{B}) = \emptyset$
    \\$\forall x (x \in A \rightarrow \lnot(x \in B) \equiv \forall x (x \in A \rightarrow x \not \in \bar{B})$
    \\$\forall x (x \in A \rightarrow x \not \in \bar{B}) \equiv \forall x (x \in A \rightarrow x \in B)$
    \\$\therefore A \cap \bar{B} = \emptyset \rightarrow A \subseteq B$ (We can also use the complement laws, also used in part(a), but this time we go backwards.

\end{enumerate}

\item  (2 pts)
Show that $(p \rightarrow q) \vee (p \rightarrow r) $ and $p \rightarrow (q \vee r)$ are logically equivalent.
\begin{displaymath}
    \begin{array}{|c|c|c|c|c|}
    p & q & r & (p \rightarrow q) \lor (p \rightarrow r) & p \rightarrow (q \lor r)\\
    \hline
      T & T & T & T & T\\
      T & T & F & T & T\\
      T & F & T & T & T\\
      T & F & F & F & F\\
      F & T & T & T & T\\
      F & T & F & T & T\\
      F & F & T & T & T\\
      F & F & F & T & T

    \end{array}
    \end{displaymath}
    \\We can also see if "p implies q,   or p implies r", then "p implies q or r"
    \\$\therefore$ We can see that both $(p \rightarrow q) \lor (p \rightarrow r)$ and $p \rightarrow (q \lor r)$ are both logically equivalent.

\item (3 pts)
Translate in two ways each of these statements into logical
expressions using predicates, quantifiers, and logical
connectives. First, let the domain consist of the students
in your class and second, let it consist of all people.
\\S(x) = A student in the class
\begin{enumerate}
	\item There is a person in your class who cannot swim.
    \\SW(x) = x can swim
    \\1. (Students): $\exists x \lnot SW(x)$
    \\2. (All People): $\exists x (S(x) \land \lnot SW(x)) $\\
	\item All students in your class can solve quadratic equations.
    \\EQ(x) = x can solve quadratic equations equation
    \\1. (Students): $\forall x EQ(x)$
    \\2. (All People): $\forall x (S(x) \land EQ(x))$\\
	\item  Some student in your class does not want to be rich.
    \\R(x) = x wants to be rich
    \\1. (Students): $\exists x \lnot R(x)$
    \\2. (All People): $\exists x (S(x) \land \lnot R(x))$\\
\end{enumerate}

\item (3 pts)
Express the negations of these propositions using quantifiers,
and in English.
\begin{enumerate}
	\item Every student in this class likes mathematics.
    \\Original proposition: (M(x) = people who like math)
     $\forall x M(x)$
     \\Negation: $\exists x \lnot M(x)$
     \\Negation in English: It is the case that not all students like mathematics.\\

\item  There is a student in this class who has never seen a computer.
    \\Original proposition: (C(x) = people who have seen a computer) $\exists x \lnot C(x)$
    \\Negation:
    \\$\lnot (\exists x \lnot C(x)) \equiv \forall x \lnot(\lnot (C(x))$
    \\$\equiv \forall x C(x)$
    \\Negation in English: All students in the class have seen a computer.\\

\item There is a student in this class who has taken every
mathematics course offered at this school.
\\Original proposition: (MC(x) = people who have taken all math courses) $\exists x MC(x)$
\\Negation: $\forall x \lnot MC(x)$
\\Negation in English: All students in this class have not taken all the math courses offered.\\

\end{enumerate}
\item (2 pts)
\begin{enumerate}
	 \item Suppose that $Q(x)$ is the statement $x + 1 = 2x.$ What are the truth values of $\forall x Q(x)$ and $\exists x Q(x)$?
\\For $\forall Q(x)$:
\\The truth value for this statement will be false since not all values will allow $x + 1 = 2x$. Take for example x = 4; $4 + 1 = 2(4)$; $5 \neq 8$. $\therefore$ this statement is false since not all values of x satisfies the statement $x + 1 = 2x$.\\
\\For $\exists Q(x)$:
\\The truth value for this statement will be true since there is a value that exists that will satisfy the statement $x + 1 = 2x$. If we let x = 1; $1 + 1 = 2(1)$; 2 = 2. $\therefore$ This statement is True.
	\\
	\item Let $P(m, n)$ be "$n$ is greater than or equal to $m$" where the domain (universe of discourse) is the set of
	nonnegative integers. What are the truth values of $\exists n \forall m P(m, n)$ and $\forall m\exists n  P(m, n)$?

For $\exists n \forall m P(m, n)$:
\\This equation returns FALSE. If we read it, it says "For some n, all m are such that P(m,n)". For some nonnegative integer n, there is any number m, such that $m < n$. This is false since there is no number n that is greater than all the numbers m.\\

For $\forall m \exists n P(m, n)$:
\\This equation returns TRUE. If we read it, it says "For all m, there is a n such that P(m,n)". For any nonnegative integer m, there is some n that $m < n$.



\end{enumerate}

\end{enumerate}
\end{document} 