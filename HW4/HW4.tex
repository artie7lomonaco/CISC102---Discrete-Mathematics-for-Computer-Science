\documentclass[12pt]{article}
\title{CISC 102 (Fall 20)\\ Homework \#4: Functions $\;$   (20 Points) }
\author{Student Name/ID:. . . . . . . .}
\date{}


\usepackage[left=2cm, right=3cm, top=1cm]{geometry}

\usepackage{amsmath}
\usepackage{amsfonts}
\usepackage{amssymb}
\usepackage{amsthm}
\usepackage{bm}						% for \bm -- bold almost everywhere
\usepackage{mdwlist}					% for {itemize*}
\usepackage{enumerate}
\usepackage{xcolor,graphicx}
\usepackage{relsize,etoolbox}
\usepackage{tikz}

\newcommand{\abs}[1]{\left| #1 \right|}
\newcommand{\ab}[1]{\left[ #1 \right]}
\newcommand{\rb}[1]{\left( #1 \right)}
\newcommand{\set}[1]{\left\{ #1 \right\}}
\newcommand{\norm}[1]{\left\| #1 \right\|}
\newcommand{\wt}[1]{\widetilde{#1}}
\newcommand{\pdif}[2]{\frac{\partial #1 }{\partial #2}}
\newcommand{\pdifm}[3]{\frac{\partial^{#3} #1 }{\partial #2^{#3}}}

\renewcommand{\a}{\ensuremath{\mathcal{A}}}
\renewcommand{\b}{\ensuremath{\mathcal{B}}}
\newcommand{\bb}{\ensuremath{\mathbb{B}}}
\newcommand{\ee}{\ensuremath{\mathbb{E}}}
\newcommand{\f}{\ensuremath{\mathcal{F}}}
\newcommand{\g}{\ensuremath{\mathcal{G}}}
\newcommand{\h}{\ensuremath{\mathcal{H}}}
\renewcommand{\l}{\ensuremath{\mathcal{L}}}
\newcommand{\map}{\longrightarrow}
\newcommand{\nn}{\ensuremath{\mathbb{N}}}
\newcommand{\one}{\ensuremath{\mathbf{1}}}
\newcommand{\p}{\ensuremath{\mathcal{P}}}
\newcommand{\pp}{\ensuremath{\mathbb{P}}}
\newcommand{\qq}{\ensuremath{\mathbb{Q}}}
\newcommand{\rr}{\ensuremath{\mathbb{R}}}
\newcommand{\zz}{\ensuremath{\mathbb{Z}}}

\let\oldemptyset\emptyset
\let\emptyset\varnothing

\newcommand{\var}{\mathrm{var}}

\DeclareMathOperator{\argmax}{argmax}

\begin{document}

\maketitle

\par\noindent Solutions are due before 11:59 PM on \textbf{October 20, 2020 }.


\begin{enumerate}

\item (2pts) \\
Determine whether the mappings from   $ \mathbb{R}$ to $ \mathbb{R}$ shown below  are or are not functions, and explain your decision.
\begin{enumerate}
	\item $f(x) = 1/x$
    \\This function is not a function from $\mathbb{R}$ to $\mathbb{R}$ since the function is not defined at $f(0) = 1/0$.
	\item $f(x) = \sqrt{x}$
    \\This function is also not a function from $\mathbb{R}$ to $\mathbb{R}$ since $f(x) = \sqrt{x}$ is only defined for nonnegative values of x and will only return their square root which will be positive. $\therefore$ this function would be a function of the domain and codomain of $\mathbb{R}^+$.
\end{enumerate}


\item  (2pts) \\Determine whether each of the following functions from $\mathbb{R}$ to $\mathbb{R}$ is a bijection, and explain your decision. HINT: Plotting these functions may help you with your decision.
\begin{enumerate}
	\item $f(x) = -x^2 + 2$
    \\This function is not a bijection sine it is neither onto or one-to-one. It isn't one to one since, take for example and $x = 1$ and $x = -1$. $-(1)^2 + 2 = -(-1)^2 + 2 = 1$. It is also not onto since of the domain $\mathbb{R}$ to $\mathbb{R}$, the range of $f(x)$ is all the values that are $\{\,f(x)\mid f(x) \leq 2\,\}$. $\therefore$ this function is not a bijection.
	\item  $f(x) = x^3 - x^2$
    \\This function is not a bijection since it is onto, but not one-to-one. First off this function is onto since the function is defined everywhere from $\mathbb{R}$ to $\mathbb{R}$. On the other hand it is not one-to-one since some values of $f(x)$ are repeated. If we factor the function it will look like $x^2(x - 1)$, and if we look at those two factors, at $f(x) = 0$, $x = 0$ and $x = 1$.$\therefore$This function is not a bijection since it is only onto but not one-to-one.
\end{enumerate}

\item  (2pts) \\Suppose the function $f : A \rightarrow B$ is a bijection. What can you say about the values $|A|$ and $|B|$?
    \\We can say that both $A$ and $B$ have the same cardinality. Since if it is one-to-one and onto, all the values of $a \in A$ have a unique value of $b \in B$ and vice versa.Take for example a function that is a bijection (since it can also be inverted) in the form of $f(x) = x^3$ from the domain and codomain of $\mathbb{R}$ to $\mathbb{R}$. Its range is $\mathbb{R}$, and it meets the condition stated above.$\therefore$We can conclude that the cardinality of A and B have to be equal for the function $f : A \rightarrow B$ to be a bijection.

\item  (2pts) \\Let \(f:\lbrace{1,2,3,4,5,6}\rbrace \rightarrow \lbrace{red,yellow,beige, green,umber,teal}\rbrace\) be a one-to-one function. Prove, by contradiction,  that \(f\) is a bijection.
    \\If we prove this by contradiction then we are assuming that the function is not a bijection. For the function to not be a bijection it must not be onto, one-to-one, or neither. But since this function is already a one-to-one, and the cardinality of both sets is equal, then each element of set $\{1,2,3,4,5,6\}$ will have a unique element in set $\{red,yellow,beige, green,umber,teal\}$. With this we have proven that this function is one-to-one, and also onto, proving the contradiction false.

\item  (2pts)
Let \(A = \lbrace{1,2,3,4}\rbrace\) \\
Let \(B = \lbrace{a,b}\rbrace\) \\
Let \(C = \lbrace{curling,hockey,table\text{-}tennis}\rbrace\)
\begin{enumerate}
\item  How many {\it  one-to-one } functions are there from \(C\) to \(A\)?
\\From C to A we see that C has 3 elements and A has 4 elements. The first element, "curling", can be mapped out to any of the elements in A which is 4. The second element, "hockey", Has to be mapped out to the 3 remaining elements. And finally, "table\text{-}tennis", the last element, has to be mapped out to the last 2 remaining elements of A. If we multiply this out, $2 \times 3 \times 4 = 24$. $\therefore$there are 24 different one-to-one functions from C to A.
\item  How many {\it  onto }   functions are there from \(C\) to \(B\)?\\
(Hint: count the non-onto functions)
\\For this function from C to B, there is a total number of functions of $2^3=8$. And if we are looking for the non-onto functions, in which still all the elements of C are mapped to one element in B, then there are only 2 non-onto functions from C to B. So now for the total number of onto functions from C to B is $8 - 2 = 6$. $\therefore$ there are 6 onto functions from C to B.
\\
\end{enumerate}

\item  (4pts) \\Decide for each of the following expressions: Is it a function? If so,

\centerline{
	\begin{tabular}{l@{\qquad}l}
		(i) what is its domain, codomain, and image? & (ii) is it injective? (why or why not)\\
		(iii) is it surjective?  (why or why not)&(iv) is it invertible?  (why or why not)
\end{tabular}}
\begin{enumerate}
	\item $f:$  $\mathbb{R}$  $\rightarrow$  $ \mathbb{R}$  defined by  $x \mapsto x^3$
    \\First off this relation is a function.
    \\(i) Domain = $\mathbb{R}$
    \\   Codomain = $\mathbb{R}$
    \\   Image = $\mathbb{R}$
    \\(ii) This function is injective since each value from the domain has a unique value of the codomain.
    \\(iii) Like said in (ii), each value in the domain has its own unique value in the codomain, which also implies the opposite.
    \\(iv) this function is invertible since it is both injective and surjective, which means in is bijective.
	\item $f: \mathbb{R} \times \mathbb{Z} \rightarrow \mathbb{Z} \textrm{ defined by }   (r,z) \rightarrow \lceil r \rceil * z$
\\(i)Domain = $\mathbb{R} \times \mathbb{Z}$
\\Codomain = $\mathbb{Z}$
\\Image = $\mathbb{Z}$
\\(ii) This function is not injective. Lets say r is any number greater than 0 but less than 1, and z is 1. If that's the case then the image of all these values will be 1, since the ceiling function returns the next greatest integer from a rational number. Therefore many values of r (in our example between grater than 0 and less than 1 and z = 1), the image of this will always be 1.
\\(iii) This function is surjective, since the function can equal any value in the codomain if the values of r and z are the correct ones.
\\(iv) Since this function is not injective, and is surjective, it is not invertible since it needs to be both injective and surjective.
\end{enumerate}



\item  (2pts)\\ Let $f$ be a function from the set $A$ to the set $B$. Let $S$ and $T$ be subsets of $A$. Show that
\begin{enumerate}
	\item $f(S \cup T) = f(S) \cup f(T)$:
    \\Let $f(y) = x$
    \\(1) $x \in f(S \cup T)$
    \\(2) $y \in S \lor y \in T$
    \\(3) $f(y) \in f(S) \lor f(y) \in f(T)$ ($f(s)$ has all elements that are images of element of S)
    \\(Same with respect to $f(T)$)
    \\(4) $x \in f(S) \lor x \in f(T)$
    \\(5) $x \in f(S) \cup f(T)$
    \\(6) $f(S \cup T) \subseteq f(S) \cup f(T)$
    \\(7) $x \in f(S) \lor \in f(T)$
    \\(8) $y \in S \lor y \in T$
    \\(9) $y \in S \cup T$
    \\(10) $f(y) \in f(S \cup T)$
    \\(11) $x \in f(S \cup T)$
    \\(12) $f(S) \cup f(T) \subseteq f(S \cup T)$
    \\$\therefore$ since $f(S) \cup f(T) \subseteq f(S \cup T)$ and $f(S \cup T) \subseteq f(S) \cup f(T)$
    \\$\therefore f(S \cup T) = f(S) \cup f(T)$
	\\\item $f(S \cap T) \subseteq f(S) \cap f(T)$.
    \\Let $f(y) = x$
    \\(1) $x \in (S \cap T)$
    \\(2) $y \in S \land y \in T$
    \\(3) $f(y) \in f(S) \land f(y) \in f(T)$
    \\(4) $x \in f(S) \land x \in f(T)$
    \\(5) $x \in (f(S) \cap f(T))$
    \\(6) $f(S \cup T) \subseteq f(S) \cup f(T)$
    \\(7)
\end{enumerate}

\item  (1pts) find the inverse function of $f(x)=x^3+1$
\\$y = x^3 + 1 \rightarrow y - 1 = x^3 \rightarrow \sqrt[3]{y - 1} = x$
\\$\therefore f^{-1} (y) = \sqrt[3]{y - 1}$

\item  (2pts) \\
Suppose that $g$ is a function from $A$ to $B$ and f is a function
from $B$ to $C$.
\begin{enumerate}
	\item Show that if both f and g are one-to-one functions,
	then $f \circ g$ is also one-to-one.
    \\$f$ is one-to-one if $f(x) = f(y)$, then $x = y$
    \\$g$ is one-to-one if $g(x) = g(y)$, then $x = y$
    \\$f(g(a)) = f(g(b))$
    \\Since $f$ is one-to-one: $g(a) = g(b)$
    \\Since $g$ is one-to-one: $a = b$
    \\$\therefore$ this proves that $f \circ g$ is one-to-one.
	\item Show that if both $f$ and $g$ are onto functions, then $f \circ g$
	is also onto.
    \\$f$ is onto if $\forall c \in C \exists b \in B$: $f(b) = c$
    \\$g$ is onto if $\forall b \in B \exists a \in A$: $g(a) = b$
    \\Let $f(y) = x $ and $g(z) = y$
    \\$(f \circ g)(z) = f(g(z)) = f(y) = x$
    \\$\therefore$ $f \circ g$ is onto.
\end{enumerate}

\item  (1pts) \\
Find  $f \circ g$  and  $g \circ f$, where $f(x) = x^2 + 1$ and $g(x) = x + 2$,
are functions from $\mathbb{R}$ to $\mathbb{R}$.
\\$f \circ g = (x + 2)^2 + 1 = x^2 + 4x + 4 + 1 = x^2 + 4x + 5$
\\$\therefore f \circ g = x^ + 4x + 5$
\\$g \circ f = (x^2 + 1) + 2 = x^2 + 3$
\\$\therefore g \circ f = x^2 + 3$


%%%%%%%%%%%%
\end{enumerate}
\end{document} 