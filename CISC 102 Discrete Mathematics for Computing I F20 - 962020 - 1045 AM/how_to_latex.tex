\documentclass[12pt, answers]{exam}
\usepackage{setspace}
\usepackage[utf8]{inputenc}
\usepackage{mathtools, amssymb, amsthm}
\pagestyle{plain}

% \newtheorem{problem}{Problem}
% \theoremstyle{definition}
% \newtheorem*{solution}{Solution}

%%%%%%%%%%%%%%%%%%%%%%%%%%%%%%%%%%%%%%%%%%%%%
% LEAVE EVERYTHING ABOVE THIS LINE ALONE!!! %
%%%%%%%%%%%%%%%%%%%%%%%%%%%%%%%%%%%%%%%%%%%%%

\usepackage[colorlinks]{hyperref} % you can delete this when you write your documents...

\title{\LaTeX{} Example Document}  % your title goes here
\author{Chris Eppolito}            % your name goes here
\date{15 March 2020}               % the date goes here

\begin{document}
\maketitle     % leave this alone
\doublespacing % leave this alone

This is a sample document to show you how to use \LaTeX{} to write nice documents for your homework submission.  I have built the document to get you started.

Notice that the text is double-spaced.  This is on purpose!  I expect you to submit the PDF file to me so I can mark it with comments and return it to you.

\begin{questions}

\question This is a problem.  It deserves to be stated properly, and then solved below!

Once you have stated a problem, you can solve it below using the \verb@solution@ environment.

\begin{solution}
This is my solution to the first question.  When I write maths, I always enclose it between \verb@\( CONTENTS \)@.  This way \LaTeX{} knows I mean to write in-line maths.  Something like \(e^{i\pi} + 1 = 0\) is in-line maths.

To write display-style maths, I enclose it between \verb@\[ CONTENTS \]@.  This way \LaTeX{} knows I mean to write display-style maths.  Something like \[e^{i\pi} + 1 = 0\] is display-style maths.

For multi-line display-style maths, you will need to make friends with \verb@align*@.  This environment uses the symbol \verb@&@ for alignment, and uses \verb@\\@ for line-breaks.  Have a look at the code for the following to see what I mean.
\begin{align*}
    S &= \{n \in \mathbb{N} : n + 5 \leq 50\} \\
    &= \{n \in \mathbb{N} : n \leq 45\} \\
    &= \{n \in \mathbb{Z} : 0 \leq n \leq 45\}
\end{align*}
Notice that the symbols \verb@\@ and \verb@{@ and \verb@}@ are special in \LaTeX{}.  Every command begins with a \verb@\@, and the \texttt{.tex} file won't compile if you have unbalanced braces because they are used to determine limited scopes.  To get the \(\backslash\) for set difference, type \verb@\setminus@, and to get the braces for sets, type \verb@\{@ and \verb@\}@.

Some commands only make sense in math-mode (i.e.~in one of the \verb@\(\)@, \verb@\[\]@, or \verb@\begin{align*}\end{align*}@ environments).  To make subscripts use an underscore \verb@_@, and to make superscripts use a caret \verb@^@.  These are treated differently depending on the math-mode!  The command \verb@\sum_{k = 0}^{n} \binom{n}{k} = 2^n@ looks like \(\sum_{k = 0}^{n} \binom{n}{k} = 2^n\) in-line, but like so in display-style.
\[
\sum_{k = 0}^{n} \binom{n}{k} = 2^n
\]

To end your proofs with a tombstone (like I do in class), write \verb@\qed@ at the end.\qed
\end{solution}

\end{questions}

Using \href{https://www.overleaf.com/}{Overleaf} makes writing \LaTeX{} fairly straightforward.  It comes with many helpful features.  If you run into problems, search on Overleaf for the answer.

Using \href{http://detexify.kirelabs.org/classify.html}{Detexify} is good for finding symbols (e.g.~\(\gamma\) or an \(\alpha\)).  There you just handwrite the symbol you want, and the site pattern-matches your symbol to find the command.

\end{document}

%%% Local Variables:
%%% mode: latex
%%% TeX-master: t
%%% End:
